\documentclass[12pt, a4paper]{article}
\usepackage{graphicx} % Required for inserting images
\usepackage{amsmath}% Para a região de matemática.
\usepackage{hyperref} % para usarmos links
\graphicspath{{images/}} %configurando o pacote de graphicx, aqui referenciamos para o código onde nosso banco de imagens está! Por exemplo, nesta linha, temos o banco de imagens referenciado na pasta "Images"

\title{Aprendendo Latex}
\author{Faezinn\thanks{Wytler me fez aprender.}}
\date{September 2023}

\begin{document}

\maketitle

\tableofcontents

\section{Introduction}

Primeiro documento de teste do latex. É engraçado o fato de só poder tacar as palavras aqui dentro. 

% no Latex, comentamos usando o %. É meio estranho, mas funciona!

- Alguns comandos básicos do LATex:

\begin{itemize}
    \item \textbf{Negrito:} utilizamos o negrito pelo comando textbf
    \item \textit{Italico:} utilizamos o itálico pelo comando textit
    \item \underline{Underline:} utilizamos a função underline pelo comando underline
    \item Podemos também utilizar o \emph{emph} para enfatizar algo.

\end{itemize}

\section{Imagens}

O uso de imagens é um pouco complicado.

Vou tentar aplicar o quê descobri estudando:

\begin{figure}[h]
    \centering % centraliza a imagem no documento.
    \includegraphics[width=0.30\textwidth]{Images/naruto feliz.jpeg} % inclui a imagem e organiza o tamanho dela.
    \caption{Naruto aprova este documento} % cria uma legenda para a imagem
    \label{fig:naruto feliz}
\end{figure}

% \includegraphics{Images/naruto feliz.jpeg} aqui, escolhemos a imagem no pacote images.

Usando \textbf{includegraphics} é possivel colocar imagens no documento.

Mas criar uma \textbf{área} para a figura é mais efetivo, usando begin{figure} e adicionando as informações de interesse, como tamanho, legenda e nome da imagem. 

\ref{fig:naruto feliz}.narutofeliz % referencia o numero da figura.

\section{Listas}

É possível criar listas utilizando os seguintes comandos:
\begin{enumerate}
    \item Utilizando begin{\textit{Alguma-coisa}}para começar o local da lista
    \item Utilizando \textit{Enumerate} como argumento do begin para numerar os itens da lista
    \item Utilziando \textit{Itemize} no argumento para listar com a "bolinha"
\end{enumerate}

\section{Utilizando Matemática}

Na matemática do LATEX, podemos utilizar vários comandos para fazer o "typeset":

\begin{itemize}
    \item Utilizando \verb|\(texto\)|, que gera:
        \begin{enumerate}
            \item \(E = M*C^2\)
            \item \(A^2 = B^2 + C^2\)
        \end{enumerate}
    \item Utilizando \verb|$texto$|, que também gera:
        \begin{enumerate}
            \item $E = M*C^2$
            \item $A^2 = B^2 + C^2$
        \end{enumerate}
    \item Ou utilizando begin(math), que também gera:
        \begin{enumerate}
            \item \begin{math}
                E = M*C^2
            \end{math}
            \item \begin{math}
                A^2 = B^2 + C^2
            \end{math}
        \end{enumerate}  
\end{itemize}
Para centralizar a equação, utilizamos begin(equation), que gera:
\[E = M*C^2\]

\subsection{Termos Importantes}
Existem milhares de termos de matemática que podem ser listados e estudados para LATEX, como:
    \begin{enumerate}
        \item Numeros/variáveis evelados, $A^B$, ou sublinhados, $A_B$, de outros números/variáveis.
        \begin{itemize}
            \item Que podem ser combinados, como no caso:
            \[T^{i_1 i_2 \dots i_x}_{j^1 j^2 \dots j^y} = 0\]
        \end{itemize}
        \item Escrevemos integrais com \verb|$\int$| e frações com \verb|$\frac{a}{b}$|
        \begin{itemize}
            \item Os limites da integral podem utilizar do item \textbf{1.} para serem definidos, como em:
            \[ \int_0^2 \frac{dx}{dt} = \frac{a^2}{b^2}\]
        \end{itemize}
        \item Também podemos definir raizes quadradas usando sqrt, como em:
        \[\sqrt{A + B} = C\]
    \end{enumerate}

\section{Estruturas básicas do Documento}
Artigos científicos normalmente utilizam várias estruturas, então aqui está algumas delas:
\begin{abstract}
    Aqui, em \textbf{Abstract}, normalmente temos um resumo geral do assunto. Algo breve, que fale sobre o documento como um todo.
\end{abstract}

Depois do resumo, é possível começar o primeiro parágrafo. Pressione "enter" duas vezes para começar o segundo.

Aqui, por exemplo, temos o segundo parágrafo.

E aqui começamos o terceiro. Você pode utilizar \verb|\\| para \\ realizar uma quebra de linha, mas continuar no mesmo parágrafo. Em adição, a função \verb|\Newline| \newline também pode ser utilizada.

\section{Criando Tabelas}
Podemos criar uma simples tabela como esta:
\begin{center}
\begin{tabular}{c c c}
 cell1 & cell2 & cell3 \\ 
 cell4 & cell5 & cell6 \\  
 cell7 & cell8 & cell9    
\end{tabular}
\end{center}
Utilizando os comandos \verb|\begin{center}|, acompanhados de \verb|\begin{tabular}{c c c}|, onde "c c c" significa o uso de três colunas, e, por fim, utilizando "nome-célula$1$" \verb|&| "nome-célula$2$" \dots \space para criar os valores de cada célula.

Para criar uma tabela com bordas e limitações, como em:
\begin{center}
\begin{tabular}{|c|c|c|} 
 \hline
 cell1 & cell2 & cell3 \\ 
 \hline
 cell4 & cell5 & cell6 \\ 
 \hline
 cell7 & cell8 & cell9 \\ 
 \hline
\end{tabular}
\end{center}
Podemos utilizar \textbf{uma barra reta} entre os "c's" para definir a divisão das colunas, e \verb|\hline| para as linhas horizontais.

\section{Bibliografia importante}
Alguns links úteis para depois:
\begin{itemize}
    \item \href{https://www.overleaf.com/learn/latex/Mathematical_expressions}{Expressões matemáticas em LATEX}
    \item \href{https://www.overleaf.com/learn/latex/List_of_Greek_letters_and_math_symbols}{Lista de letras gregas e símbolos matemáticos}
    \item \href{https://www.overleaf.com/learn/latex/Paragraphs_and_new_lines}{Parágrafos e Linhas}
    \item \href{https://www.tablesgenerator.com/}{Gerador de imagens em Latex}
\end{itemize}

\end{document}
